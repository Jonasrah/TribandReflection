\section{Reflection}
The following sections will contain my most noteworthy reflections during my internship at Triband.
\subsection{Agility}
Triband work fast. \citeA{buxton} describes sketching as being quick to make, can be made when needed, the cost of it is not prohibiting, disposable once its use is depleted and it is plentiful in quantity. These are all characteristics that can describe the current levels in Golf All the Things. The creation of a level starts with a wacky idea, it is written on a post-it and pinned to the "New" board. The post-its are only descriptive in a minimal sense and the work only really begins inside the game engine. The perhaps most used sketching application is in the medium of pen and paper, however, I would argue that because of a high expertise in the toolset for creating games, Triband are using these tools as a sketching method. As of writing, Golf All the Things has over 200 levels, whereas about 20\% is expected to make it to the final game. I really appreciate this approach to game design since it requires the designer to really pinpoint what the interesting thing about his/her idea is, and after it is implemented, move on. Then, as a second phase the game designers can go through the sketches and decide which ones to develop further. Of course, this method requires a high amount of practicing and cannot be adopted by novices, but in my opinion it is a great method both for the development of games but also as an initial exploration of different game ideas. One that is definitely worth the effort of learning tools such as rough 3D modelling, illustration, game engines and programming.
\subsection{Programming as a Design Tool}
Being in a small studio, no one really had time to assist in the creation of prototypes other than advice. This meant that to actually create a prototype I had to program. I have taken programming courses outside of my study program, so I had some experience before my internship started, but had I not had this, I would not have been able to design any levels. In the near future, after graduating, I expect this to be relevant since I am not expecting to get hired in a position where I am able to assign design tasks without at least being able to complete these tasks myself. While I might be able to acquire such a position eventually, I am expecting my programming skills to be of importance on the other side of graduating.

Another argument for programming as a design tool is the inspiration that can come from bugs. When I program I rarely get the desired solution in the first run. This means that situations arise that I would not have thought of myself, and this can, in turn, lead to a more interesting solution or at least a higher degree of certainty in your desired solution. A concrete example of this is the shooter control for the domino piece in my domino level. My first iteration of this was a shooter control that applied the right balance of force and rotation force for the piece to somewhat move in the player's desired direction, but it felt clunky. This lead me to the trail of thought of, yes, perhaps a domino would be like a fish out of water as soon as it is taken out of a usual domino context, which in turn lead me to think, no, the domino should not control like a fish, but instead utilise its abnormal shape in an optimal way. And thus, the final iteration was made with me feeling a higher amount of affirmation than I would have otherwise felt, had I come up with the final iteration immediately.

A final justification for designers learning to program: You can create your own tools. When I was creating the domino levels I needed a tool for aligning pieces in the game world. With programming, I could. This example did end up being probably the most advanced thing I have programmed, testifying that some experience is required to use programming in this way.
